\documentclass[12pt, preprint]{aastex}

% words
\newcommand{\project}[1]{\textsl{#1}}
\newcommand{\thecannon}{\project{The~Cannon}} 
\newcommand{\apogee}{\project{APOGEE}}
\newcommand{\corot}{\project{Corot}}
\newcommand{\kepler}{\project{Kepler}}
\newcommand{\gaia}{\project{Gaia}}
\newcommand{\most}{\project{MOST}}
\newcommand{\documentname}{\textsl{Article}}

% math
\newcommand{\numax}{\nu_{\max}}
\newcommand{\deltanu}{\Delta\nu}

\begin{document}

\title{Spectral signatures of asteroseismological quantities}
\author{M.~Ness\altaffilmark{1},
David~W.~Hogg\altaffilmark{1,2,3},
H.-W.~Rix\altaffilmark{1},
\textbf{others}}
\altaffiltext{1}{Max-Planck-Institut f\"ur Astronomie, K\"onigstuhl 17, D-69117 Heidelberg, Germany}
\altaffiltext{2}{Center for Cosmology and Particle Physics, Department of Phyics,
             New York University, 4 Washington Pl., room 424, New York, NY, 10003, USA}
\altaffiltext{3}{Center for Data Science, New York University, 726 Broadway, 7th Floor, New York, NY 10003, USA}
% \altaffiltext{4}{NSF Astronomy and Astrophysics Postdoctoral Fellow}
% \altaffiltext{5}{Department of Physics \& Astronomy, Johns Hopkins University, Baltimore, MD, 21218, USA}
\email{ness@mpia.de}

\begin{abstract}%
With \thecannon, we have demonstrated that it is possible to use a small
training set of stars with noisy spectral data and known stellar-parameter labels
to build a data-driven model of stellar spectra that can be used to infer
stellar-parameter labels for other stars (with differently noisy spectral data).
Here we train this system with---instead of standard stellar parameter labels---%
asteroseismic quantities ($\numax$ and $\deltanu$) obtained from a training set of
stars with both \kepler\ observations and \apogee\ infrared spectral data.
We use cross-validation to show that
the asteroseismic labels do a good job of predicting infrared spectra,
and that infrared spectra at \apogee\ quality can be used to infer asteroseismic
properties.
For a typical \apogee\ star we can infer $\numax$ and $\deltanu$ to accuracies of
XXX and YYY, respectively, using spectroscopic data \emph{alone}.
In turn, this permits us to infer stellar ages on the red-giant branch to an
accuracy of approximately ZZZ.
These findings provide context for future asteroseismology observing efforts;
in particular, medium-resolution infrared spectroscopy might be as
effective as high-cadence time-domain photometry for inferring gross properties
of stellar interiors and ages, for at least some parts of the H-R diagram.
\end{abstract}

\keywords{%
keywords: incomplete!
---
methods: data analysis
---
methods: statistical
---
stars: abundances
---
stars: fundamental parameters
---
surveys
---
techniques: spectroscopic
}

\section{Introduction}\label{sec:Intro}

Asteroseismology surveys, such as \most, \corot, and \kepler, have
been extremely successful and productive in bringing us information
about stellar interiors and stellar parameters and ages.
These missions operate by taking high-cadence, high-precision stellar
photometry, in which stellar oscillation modes are visible in the
Fourier domain.
These missions have operated by taking long stretches of
uninterrupted, uniform, dense imaging data on single stars.
They are expensive missions, but absolutely critical to calibrate
physical stellar models and set standards for stellar parameter
estimation, all of which is required for the ultimate success of the
next generation of stellar surveys, particularly including the
\gaia\ Mission.
If \gaia\ is able to obtain stellar parameters that are both accurate
and precise, it will revolutionize our view of the Milky Way and its
formation.

In this \documentname\ we take a look at the combination of
asteroseismological and spectroscopic data for stellar parameter
estimation.
Our interest is in whether the power of asteroseismological
measurements made on a small subset of stars can be amplified to help
with parameter estimation on a much larger set, even those for which
no direct asterosiesmological measurements are possible.

Our motivation is three-fold:
For one, we are motivated by the fact that spectral data tell you
different things about stellar parameters than do asteroseismological
measurements.
For example, we expect asteroseismology to be much more sensitive
to the mean stellar density ($G\,M\,R^{-3}$), while we expect spectroscopy to be much
more sensitive to surface gravity ($G\,M\,R^{-2}$).
These complementary sensitivities indicate that a combination of
asteroseismology and spectroscopy must be more powerful on stellar
structure than either taken alone.

For two, we are motivated by the fact that it is difficult to imagine
a near future in which nearly as many stars have good-quality
asteroseismological measurements as will have good-quality
spectrscopic measurements.
The reasons for this are cultural and technical, but suffice it to say
that there are currently millions of stars targeted for
medium-resolution to high-resolution spectroscopy, but only thousands
targeted for asteroseismological measurements.
This means that \emph{if} we are going to benefit maximally from the
asteroseismology, we have to find a way to transfer it from the thousands
of direct measurements to the millions of stars that will get no direct
measurements.

For three, we are motivated by the limitations of current physics-based
stellar models.
If we had much better models of stellar interiors and exteriors, we would be
able to confidently infer physical properties of stars directly from our
asteroseismological and spectroscopic measurements.
However, we do not have such models.
In particular, physics-based models of stellar spectra have small issues that
lead to variations in stellar-parameter inferences that depend
significantly on the choice of spectral wavelength coverage, spectral
resolution, and the settings of physical nuisance parameters or
numerical modeling scheme.

This last motivation led us to develop \thecannon, a data-driven model
for stellar spectra.
DWH: This project...

\end{document}
